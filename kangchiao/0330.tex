\documentclass{article}
\usepackage{amssymb}
\usepackage{hyperref}
\usepackage{fontspec}
\usepackage{amsmath} 
\usepackage{graphicx}
\usepackage{nopageno}
\usepackage{courier}
\usepackage{listings}
\usepackage{titlesec}
%\usepackage{ulem}
\usepackage{calrsfs}
\DeclareMathAlphabet{\pazocal}{OMS}{zplm}{m}{n}
\usepackage{pgfplots}
%\usepackage{CJK}
\usepackage{xeCJK}
\setmainfont{Liberation Serif}
\setmonofont{Hack}
%\setmainfont{AR PL UKai TW}
\setCJKmainfont{AR PL UKai TW}
\renewcommand{\ttdefault}{pcr}
\usepackage{mathtools}
\DeclarePairedDelimiter\ceil{\lceil}{\rceil}
\DeclarePairedDelimiter\floor{\lfloor}{\rfloor}
\XeTeXlinebreaklocale "zh"
\usepackage[a4paper, total={6in, 9in}]{geometry}
\pagestyle{plain}
\setlength{\parskip}{5pt}
\fontsize{14pt}{20pt}\selectfont
\titleformat*{\subsection}{\Large\bfseries}
\makeatletter
\newenvironment{xitemize}{%
    \@nameuse{fontitem\romannumeral\the\@itemdepth}
    \itemize
}{%
    \enditemize
}
\makeatother
\newcommand{\fontitem}{\Large}
\newcommand{\fontitemi}{\normalsize}
\newcommand{\fontitemii}{\normalsize}

\begin{document}
%\begin{center}
%    {\Huge 2017 Winter Vacation Homework}
%\end{center}

{\huge NPSC Team}\\

\noindent{\Large Story}\\
    NPSC (National Problem Solving Contest) 是一個由台灣大學主辦的程式設計競賽。其賽制為兩個人一隊。身為一個資訊科的老師,你想要從眾多學生當中,選出兩位參加。於是你事先辦了一場比賽來作為選拔的依據。由於你認為這個比賽的默契很重要,甚至比個人的能力還重要,因此,你的標準並不是高分與否,而是要求「成績相同」。請問,在這樣的標準之下,有幾種組隊的方法?

\noindent{\Large Input}\\
    第一行包含一個正整數 $n$,表示你有幾位學生。接下來有 $n$ 個數字,分別表示那 $n$ 為同學的成績。

\noindent{\Large Output}\\
    請輸出一個數字,表示組隊的方法數。\\

\noindent{\Large Sample Input}\\
    \texttt{
        10\\
        10 1 2 2 10 3 3 3 2 5\\
    }


\noindent{\Large Sample Output}\\
    \texttt{
        7\\
    }

\newpage



{\huge NPSC Rank}\\

\noindent{\Large Story}\\
    在比賽結束後,你想要知道你的學生是第幾名。這場比賽是按照解題數量作為排名的依據。解題數越多,排名越靠前。

\noindent{\Large Input}\\
    第一行包含一個正整數 $n$ ,表示架子上有幾個商品。第二行包含 $n$ 個整數,分別表示每個商品的高度。\\
    \(1\leq n\leq 100, 0\leq\text{高度}\leq 2147483647\).\\

\noindent{\Large Output}\\
    請輸出一個整數,表示遞減商品對的對數。任兩個滿足左邊比右邊高的商品被稱作一個遞減商品對。\\

\noindent{\Large Sample Input}\\
    \texttt{
        5\\
        5 2 6 5 1\\
    }

\noindent{\Large Sample Output}\\
    \texttt{
        6\\
    }

\noindent{\Large Hint}\\
    在範例測資中,6對的遞減商品對分別為 (0,1), (0,4), (1,4), (2,3), (2,4), (3,4). 這幾對商品都滿足左邊的比右邊的高。\\
\newpage

{\huge IKEA Meatball}\\

\noindent{\Large Story}\\
    除了家具之外, IKEA 的肉丸也很有名。在出餐的時候,他們會把肉丸排成一直線,而每個肉丸都會給你若干單位的滿足度。由於某些原因,(可能是出於方便,也可能是因為你當時胃口不是太好),你只吃 $k$ 個連續的肉丸,而你好奇你最多可以獲得多少單位的滿足度。\\

\noindent{\Large Input}\\
    第一行包含一個正整數 $n$ ,表示總共有幾顆肉丸。而第二行則包含 $n$ 個整數,分別表示每顆肉丸能給你多少單位的滿足度。而最後一行則包含一個正整數 $k$ ,表示你打算吃幾顆肉丸。
    $1\leq n\leq 100000, 0\leq \text{每顆肉丸的滿足度}\leq 1000, 1\leq k\leq n$.\\

\noindent{\Large Output}\\
    請輸出一個整數,表示你吃了 $k$ 顆連續的肉丸之後,最多可以獲得多少單位的滿足度。\\

\noindent{\Large Sample Input}\\
    \texttt{
        10\\
        2 5 10 1 2 2 10 3 3 3\\
        5\\
    }


\noindent{\Large Sample Output}\\
    \texttt{
        25\\
    }

\noindent{\Large Hint}\\
    在範例測資中,如果你吃了第2、3、4、5、6 顆肉丸,你就可以獲得 25 單位的滿足度。\\

\noindent{\Large Bonus}\\
    請計算當 \(n=100000, k=50000\), 你的程式總共需要執行幾個指令。\\
    如果時限是一秒,你會 TLE 嗎?\\
    如果會的話,請試著想出一個能夠 AC 的作法。\\
\newpage
\end{document} 
